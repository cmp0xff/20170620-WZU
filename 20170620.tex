% $Header: /Users/joseph/Documents/LaTeX/beamer/solutions/conference-talks/conference-ornate-20min.en.tex,v 90e850259b8b 2007/01/28 20:48:30 tantau $

\documentclass{beamer}

\usefonttheme{professionalfonts}%  don't change fonts inside beamer
% This file is a solution template for:

% - Talk at a conference/colloquium.
% - Talk length is about 20min.
% - Style is ornate.



% Copyright 2004 by Till Tantau <tantau@users.sourceforge.net>.
%
% In principle, this file can be redistributed and/or modified under
% the terms of the GNU Public License, version 2.
%
% However, this file is supposed to be a template to be modified
% for your own needs. For this reason, if you use this file as a
% template and not specifically distribute it as part of a another
% package/program, I grant the extra permission to freely copy and
% modify this file as you see fit and even to delete this copyright
% notice. 


\mode<presentation>
{
  % \usetheme{Warsaw}
  % or ...

  \setbeamercovered{transparent}
  % or whatever (possibly just delete it)
}


%\usepackage[english]{babel}
% or whatever

%\usepackage[latin1]{inputenc}
% or whatever
\usepackage{polyglossia}
\setmainlanguage{english}
\usepackage{fontspec}
\usepackage{xeCJK}
\usepackage{unicode-math}
	\setmathfont{Latin Modern Math} % default
	%\setmathfont[range=\mathalpha]{Asana Math}
	\setmathfont{Asana Math}[range={\mathbin}] %\mathord
	\setmathfont{STIX Math}[range={"02609}] % ☉
	\setmathfont{XITS Math}[range={"1D4B6-"1D4CF}] % Script, Latin, lowercase
	\setmathfont{Latin Modern Math}[range={"1D608-"1D63B}, sans-style=italic]
	\setmathfont{Latin Modern Math}[range={
		"00391-"003A9,
		"003B1-"003F5, 
		"1D6A8-"1D6E1},	% Bold Greek
		sans-style=upright]
		
	%\setmathfont{⟨font name⟩}[range=⟨unicode range⟩,⟨font features⟩]
\usepackage{siunitx}
% ':=' as \coloneqq
%\usepackage{mathtools}
\usepackage{empheq} % numcases


%\usepackage{times}
%\usepackage[T1]{fontenc}
% Or whatever. Note that the encoding and the font should match. If T1
% does not look nice, try deleting the line with the fontenc.


\title%[Short Paper Title] % (optional, use only with long paper titles)
{An Introduction to \cite{Wang2017}}

\subtitle{An Attempt to Avoid the Old Cosmological Constant Problem}

\author[Wang] % (optional, use only with lots of authors)
{Yi-Fan Wang (王\ 一帆)}
%\inst{1} \and %S.~Another\inst{2}}
% - Give the names in the same order as the appear in the paper.
% - Use the \inst{?} command only if the authors have different
%   affiliation.

\institute[Uni zu Köln] % (optional, but mostly needed)
{
  %\inst{1}%
	Institut für Theoretische Physik\\
	Universität zu Köln}
  %\and
  %\inst{2}%
  %Department of Theoretical Philosophy\\
  %University of Elsewhere}
% - Use the \inst command only if there are several affiliations.
% - Keep it simple, no one is interested in your street address.

%\date[BCGS Admission 2017]
% (optional, should be abbreviation of conference name)
%{Admissions Academy of\\Bonn-Cologne Graduate School for Physics and 
%Astronomy\\March 30, 2017}
% - Either use conference name or its abbreviation.
% - Not really informative to the audience, more for people (including
%   yourself) who are reading the slides online

\subject{General Relativity and Cosmology}
% This is only inserted into the PDF information catalog. Can be left
% out. 



% If you have a file called "university-logo-filename.xxx", where xxx
% is a graphic format that can be processed by latex or pdflatex,
% resp., then you can add a logo as follows:

\pgfdeclareimage[height=1.0cm]{university-logo}%
{./logos/uni-koeln/UzK_Logo_ger.pdf}
\logo{\pgfuseimage{university-logo}}



% Delete this, if you do not want the table of contents to pop up at
% the beginning of each subsection:
\AtBeginSection[]
{
  \begin{frame}<beamer>{Outline}
    \tableofcontents[currentsection,currentsubsection]
  \end{frame}
}


% If you wish to uncover everything in a step-wise fashion, uncomment
% the following command: 

%\beamerdefaultoverlayspecification{<+->}

\usepackage[citestyle=alphabetic,doi=false,isbn=false,url=false,
defernumbers=true]%
	{biblatex}
\addbibresource{cosmo-const.bib}

\usepackage{tikz}
% \usepackage{tikz-3dplot}
% \usetikzlibrary{positioning,shapes,arrows}
\usetikzlibrary{decorations.pathmorphing}
\usetikzlibrary{calc}
% \usetikzlibrary{decorations.markings}

\usepackage{pgfplots}
\pgfplotsset{compat=1.13}
\usepgfplotslibrary{fillbetween}

\usepackage{cleveref}

\usepackage{braket}

% Mathematical constants
\newcommand{\ii}{{\Bbbi}}
\newcommand{\ee}{{\Bbbe}}
\newcommand{\pp}{{\Bbbpi}}

\newcommand{\lc}{\mitsansc} % speed of light in vacuum
\newcommand{\bk}{\mitsansk} % Boltzmann's constant
\newcommand{\phs}{\hslash} % reduced Planck constant
\newcommand{\ph}{\Planckconst} % Planck constant

% Bracket-like
\newcommand{\rbr}[1]{{\left(#1\right)}}
\newcommand{\sbr}[1]{{\left[#1\right]}}
\newcommand{\cbr}[1]{{\left\{#1\right\}}}
\newcommand{\abr}[1]{{\left<#1\right>}}
\newcommand{\vbr}[1]{{\left|#1\right|}}
\newcommand{\fat}[2]{{\left.#1\right|_{#2}}}
% Functions; note the space between the name and the bracket!
\newcommand{\rfun}[2]{{#1}\mathopen{}\left(#2\right)\mathclose{}}
\newcommand{\sfun}[2]{{#1}\mathopen{}\left[#2\right]\mathclose{}}
\newcommand{\cfun}[2]{{#1}\mathopen{}\left\{#2\right\}\mathclose{}}
\newcommand{\afun}[2]{{#1}\mathopen{}\left<#2\right>\mathclose{}}
\newcommand{\vfun}[2]{{#1}\mathopen{}\left|#2\right|\mathclose{}}
% Differentials
\newcommand{\Dif}{\BbbD}
\newcommand{\Diff}{\,\BbbD}
\newcommand{\dd}{\Bbbd}
\newcommand{\ddf}{\,\Bbbd}
\newcommand{\dva}{\mupdelta} % no better way?!
\newcommand{\dvar}{\,\mupdelta}
% Fraction-like
\newcommand{\frde}[2]{{\frac{\dd{#1}}{\dd{#2}}}}
\newcommand{\frDe}[2]{{\frac{\Dif{#1}}{\Dif{#2}}}}
\newcommand{\frpa}[2]{{\frac{\partial{#1}}{\partial{#2}}}}
% Equal marks
\newcommand{\eeq}{{\overset{!}{=}}}
\newcommand{\lls}{{\overset{!}{<}}}
\newcommand{\ggt}{{\overset{!}{>}}}
\newcommand{\lle}{{\overset{!}{\le}}}
\newcommand{\gge}{{\overset{!}{\ge}}}
% overline-like marks
\newcommand{\ol}[1]{{\overline{{#1}}}}
\newcommand{\ul}[1]{{\underline{{#1}}}}
\newcommand{\tld}[1]{{\widetilde{{#1}}}}
\newcommand{\ora}[1]{{\overrightarrow{#1}}}
\newcommand{\ola}[1]{{\overleftarrow{#1}}}
\newcommand{\td}[1]{{\widetilde{#1}}}
\newcommand{\what}[1]{{\widehat{#1}}}
%\newcommand{\prm}{{\symbol{"2032}}}

% Math operators
% Why does \DeclareMathOperator not work?
\DeclareMathOperator{\sgn}{sgn}
\DeclareMathOperator{\grad}{grad}
\DeclareMathOperator{\curl}{curl}
\DeclareMathOperator{\rot}{rot}
\DeclareMathOperator{\opdiv}{div}
\DeclareMathOperator{\opdeg}{deg}

\DeclareMathOperator{\sech}{sech}
\DeclareMathOperator{\csch}{csch}

\DeclareMathOperator{\diag}{diag}
\DeclareMathOperator{\tr}{tr}

\DeclareMathOperator{\ad}{ad}

\DeclareMathOperator{\expi}{expi}

% Group and Algebras
\newcommand{\SO}{\mathsf{SO}}
\newcommand{\SU}{\mathsf{SU}}
\newcommand{\so}{\mathfrak{so}}
\newcommand{\su}{\mathfrak{su}}

% Physical constants
\newcommand{\nG}{\mitsansG} % Newton's constant


\begin{document}

\begin{frame}
  \titlepage
\end{frame}

\begin{frame}{Outline}
  \tableofcontents
  % You might wish to add the option [pausesections]
\end{frame}


% Structuring a talk is a difficult task and the following structure
% may not be suitable. Here are some rules that apply for this
% solution: 

% - Exactly two or three sections (other than the summary).
% - At *most* three subsections per section.
% - Talk about 30s to 2min per frame. So there should be between about
%   15 and 30 frames, all told.

% - A conference audience is likely to know very little of what you
%   are going to talk about. So *simplify*!
% - In a 20min talk, getting the main ideas across is hard
%   enough. Leave out details, even if it means being less precise than
%   you think necessary.
% - If you omit details that are vital to the proof/implementation,
%   just say so once. Everybody will be happy with that.

\section{Overview}

\begin{frame}{Overview}
\begin{itemize}
\item If contributed by vacuum energy, the cosmological constant could
\alert{differ 120 order-of-magnitude} from observation.
\item \citeauthor{Wang2017} consider vacuum energy \alert{density} instead, 
which is subject to \alert{fluctuation}, and their result moderates the problem.
\item They assumed a localised RW metric and obtained FL-like equations for 
$\rfun{a}{t,\vec{x}}$, whose coefficient contains quantum fluctuation.
\item The equations leads to local Hubble parameters which fluctuate, but a
global average could be defined.
\item The solution to the equations is evaluated with the help of theories of
parametric oscillator and adiabatic invariance.
\item The result is qualitatively supported by numerical calculation and 
further quantitative discussions.
\end{itemize}

\end{frame}


\section{The Old Cosmological Constant Problem}

\begin{frame}{Cosmological constant and vacuum energy}
\begin{itemize}
\item GR + vacuum QFT
\begin{equation}
G_{\mu\nu} + \lambda_\text{b} g_{\mu\nu} = 8\pp\nG T^\text{vac}_{\mu\nu}
\label{eq:1}
\end{equation}
\item In Minkowski space-time, Lorentz invariance requires
\begin{equation}
T^\text{vac}_{\mu\nu} = -\rho^\text{vac}_{\mu\nu} \eta_{\mu\nu},
\end{equation}
which generalises to curved space-time as
\begin{equation}
T^\text{vac}_{\mu\nu} = -\rho^\text{vac}_{\mu\nu} g_{\mu\nu}.
\end{equation}
\item Effectively, \cref{eq:1} can be written as
\begin{align}
G_{\mu\nu} + \lambda_\text{eff} g_{\mu\nu} = 0, \qquad
\lambda_\text{eff} = \lambda_\text{b} + 8\pp\nG\rho^\text{vac};
\label{eq:Einstein-2}\\
G_{\mu\nu} = -8\pp\nG\rho^\text{vac}_\text{eff} g_{\mu\nu}, \qquad
\rho^\text{vac}_\text{eff} = \rho^\text{vac} +
\frac{\lambda_\text{b}}{8\pp\nG}
\label{eq:Einstein-3}.
\end{align}
\end{itemize}
\end{frame}

\begin{frame}{Hubble parameter and cosmological constant}
\begin{itemize}
\item Homogeneity and isotropy: Robertson--Walker metric
\begin{equation}
\dd s^2 = -\dd t^2 + \rfun{a^2}{t}\delta_{ij}\,\dd x^i\,\dd x^j.
\end{equation}
\item Hubble parameter / expansion rate $H \coloneqq \dot{a}/a$;
\cref{eq:Einstein-2,eq:Einstein-3} take the corresponding Friedmann--Lemaître
form
\begin{align}
3H^2 = \lambda_\text{eff} = 8\pp\nG\rho^\text{vac}_\text{eff},\\
3\ddot{a} = \lambda_\text{eff} a = 8\pp\nG\rho^\text{vac}_\text{eff} a.
\label{eq:FL-2}
\end{align}

\item Solution to \cref{eq:FL-2}
\begin{equation}
\rfun{a}{t} = \rfun{a}{t_0}\ee^{H \rbr{t-t_0}}
\end{equation}

\end{itemize}

\end{frame}

\begin{frame}{The Old Cosmological Problem}
\begin{itemize}

\item Contributions to $\rho^\text{vac}_\text{eff}$ or $\lambda_\text{eff}$:
vacuum fluctuation of all quantum fields, Electroweak phase transition, etc.

\item $\lambda_\text{eff}$ by vacuum fluctuation evaluated in Minkowski space:
taking neutral massless Klein--Gordon and sharp-momentum cut-off,
\begin{equation}
\left<\rho^\text{vac}\right> = \frac{\Lambda^4}{16\pp^2}.
\end{equation}
Setting $\Lambda = E_\text{P}$ results in a surpass of the observed value of
$\lambda_\text{eff}$ by $\sim 120$ orders of magnitude.

\end{itemize}

\end{frame}

\section{Matter and space-time models}

\begin{frame}{Matter: neutral massless scalar field in flat space-time}
\end{frame}

\begin{frame}{Space-time: localised Robertson--Walker metric}
\end{frame}

\section{Solving $\rfun{a}{t,\vec{x}}$: parametric oscillator}

\section{Solving $\rfun{a}{t,\vec{x}}$: adiabatic approximation}

\section{Model of Gravitation}

\begin{frame}[allowframebreaks]{Classical theory of $\rbr{1+1}$d 
Dilaton Gravity Model}{
%\cite{Callan1992,Demers1996,Ashtekar2011}
}% 1/2}

\begin{itemize}
\item The action of the dilaton gravity model reads
\begin{equation}
S = \int \dd^2 x\,\sqrt{-g}\,\cbr{\frac{\ee^{-2\phi}}{\nG}
\sbr{R+4\rbr{\nabla \phi}^2 + 4\lambda^2}
-\frac{1}{2}\rbr{\nabla f}^2},
\end{equation}
\begin{itemize}
\item $\rfun{\phi}{x}$ the dilaton field, without which topological
\item $\rfun{f}{x}$ a massless neutral scalar field representing matter
\item $\lambda > 0$ the cosmological constant
\end{itemize}

% \begin{align}
% S &= \int \dd^2 x\,\sqrt{-\ol{g}}\,\cbr{\frac{\ee^{-2\ol{\phi}}}{\nG}
% \sbr{\ol{R}+4\rbr{\nabla \ol{\phi}}^2 + 4\lambda^2}
% -\frac{1}{2}\rbr{\nabla f}^2} \nonumber \\
% &= \int \dd^2 x\,\sqrt{-g}\,\cbr{\frac{1}{\nG}\sbr{R\phi + 4\lambda^2}
% -\frac{1}{2}\rbr{\nabla f}^2},
% \label{eq:action-CGHS-dilaton-eli}
% \end{align}
% where in \cref{eq:action-CGHS-dilaton-eli}, the \alert{kinetic term} of the 
% dilaton field $\phi$ is eliminated by substituting $\phi = \ee^{-2\ol{\phi}}$ 
% and $g_{\alpha\beta} = \ee^{-2\ol{\phi}} \ol{g}_{\alpha\beta}$.
\item Has a solution which \alert{resembles} the collapsing body in 
$\rbr{3+1}$-dimensional Einstein gravitation
\end{itemize}

\begin{columns}
\begin{column}{.7\linewidth}
%\begin{center}
%\input{./graphics/graph_hawking_CGHS}
%\end{center}

\end{column}
\begin{column}{.3\linewidth}
\begin{itemize}
 \item Each point represents a \alert{point}
 \item \textbf{Thick} line: null matter \alert{shell}
\end{itemize}
\end{column}
 
\end{columns}


\end{frame}


\begin{frame}[allowframebreaks]{Quantum theory of $\rbr{1+1}$d Dilaton Gravity 
Model}{%\cite{Demers1996}
}

\begin{itemize}
\item Constraint system: Schrödinger quantisation does not apply
\item (Formally) Dirac quantisation
\begin{equation}
 \what{\mscrH}_\parallel \sfun{\Psi}{g,\phi,f} = 0, \qquad
 \what{\mscrH}_\perp  \sfun{\Psi}{g,\phi,f} = 0.
\end{equation}


%\item Apply a \alert{Born--Oppenheimer}-type approximation to 
%$\sfun{\Psi}{g,\phi,f}$ \cite[§~5.4]{Kiefer2012}

%\begin{itemize}
%\item Separate `\emph{Heavy, slow}' gravity and `\emph{light, fast}' matter
%\item Apply WKB approximation to the gravity part
%\item Assume disentanglement $\sfun{\Psi}{g,\phi,f} = \sfun{D}{g, \phi}
%\sfun{\chi}{g, \phi, f}$
%\end{itemize}

\item Semi-classical approximation: $\Psi = \ee^{\ii\rbr{\nG^{-1}S_0 + 
S_1 + \nG S_2 + \ldots}}$

\begin{itemize}
\item $\rfun{\Omicron}{\nG^{-1}}$: Hamilton--Jacobi equation for pure gravity
\item $\rfun{\Omicron}{\nG^0}$: $\Psi = \sfun{D}{g, \phi}
\sfun{\chi}{g, \phi, f}$; functional \alert{Schrödinger equation for matter}
$\ii \partial_t\sfun{\chi}{f} = \what{H}_\text{m} \sfun{\chi}{f}$, where
%\begin{align}
%,
%\label{eq:functional-sch} \\
\begin{equation}
\what{H}_\text{m} = \frac{1}{2}\int_{0}^{+\infty} \dd k\,
\rbr{-\frac{\mbfdelta^2}{\mbfdelta \rfun{f^2}{k}} + k^2 \rfun{f^2}{k}}.
\end{equation}
%\end{align}
\end{itemize}
\item A quantum field theory in curved space-time can be \alert{derived}!

\item At early time, the \alert{vacuum} wave functional is 
%(eq.\ (48))
\begin{equation}
\sfun{\chi_0}{f_\text{e}} \propto \cfun{\exp}{-\frac{1}{2} 
\int_{\BbbR^+}\dd k\, k \, \rfun{f_\text{e}^2}{k}} \sim
\prod_k \cfun{\exp}{\frac{1}{2} \frac{k}{\Lambda} \, f_\text{e}^2},
\end{equation}
while at late time it evolves to %(eq.\ (57) in \cite{Demers1996})
\begin{equation}
\sfun{\chi_b}{f_\text{l}} \propto 
\cfun{\exp}{-\int_{\BbbR}\dd p\, p \,\rfun{\coth}{\frac{\pp 
p}{2\lambda}} \vbr{\rfun{f_\text{l}}{p}}^2} \sim \prod_p \ee^{\ldots},
\label{eq:squeezed-wave-functional}
\end{equation}
where $\rfun{f_\text{e}}{k}$ and $\rfun{f_\text{l}}{p}$ are the Fourier 
transform of the matter field at early and late time, respectively.
%\begin{equation}
%\rfun{g}{k} = \int_{-\infty}^{+\infty}\dd v\,
%\frac{\ee^{-\ii k v}}{\sqrt{2\pp}} \rfun{f}{v},
%\end{equation}

\item At late time, \alert{particle-number expectations} are
\begin{equation}
\abr{\rfun{\what{n}_b}{p}}_{\chi_b} = \rbr{\ee^{2\pp\vbr{p}/\lambda}-1}^{-1},
\end{equation}
leading to a Hawking-like \alert{black-body temperature}
\begin{empheq}[box=\fbox]{equation}
T_\text{HD} \coloneqq \lambda/2\pp.
\label{eq:hawking-dilaton}
\end{empheq}

\end{itemize}



%\item the wave functional of which 
%has a semi-classical expansion (eq.\ (10) in \cite{Demers1996})
%\begin{equation}
%\sfun{\Psi}{\rho,\phi,f} = \ee^{\ii\rbr{\nG^{-1}S_0 + S_1 + \nG S_2 + \ldots}}
%\end{equation}

\end{frame}

\section{Correlator of Field Strength}

%\begin{frame}{Correlation of the Field Strength}{Introduction}
%\begin{itemize}
%\item Measures the res
%\end{itemize}


%\end{frame}

\begin{frame}{Correlation of Fourier Modes}{The discrepancy}

\begin{itemize}
\item The Fourier-mode correlators can be calculated,
\begin{equation}
\abr{\rfun{\what{f}^\dagger}{p_1}\rfun{\what{f}}{p_2}} =
\frac{1}{T_\text{HD}} \rfun{\delta}{p_1 - p_2} \cdot
\begin{cases}
\frac{1}{2} \frac{1}{q},
&\quad \text{vacuum};\\
\frac{1}{8}\frac{\tanh\frac{q}{4}}{\frac{q}{4}},
&\quad \chi_b; \\
\frac{1}{4} \frac{\coth\frac{q}{2}}{\frac{q}{2}},
&\quad \rfun{\what{\rho}_\text{BE}}{T_\text{HD}},
\end{cases}
\end{equation}
where $q \coloneqq p_1/T_\text{HD}$
\item Diagonal elements (fluctuations) are plotted
\end{itemize}

%Note that
%\begin{equation}
%\abr{\rbr{\mbfDelta\what{f}}^2} = \abr{\what{f}^2} - \abr{\what{f}}^2 = 
%\abr{\what{f}^2} .
%\end{equation}
\end{frame}

\begin{frame}{Correlation of Fourier Modes}{Fluctuation of the Fourier 
modes: diagram in log-log scale}

%\begin{center}
%\input{./graphics/graph_fluc_fourier_color}
%\end{center}

\end{frame}

\begin{frame}{Correlation of Fourier Modes}{Fluctuation of the Fourier 
modes: interpretation}

\begin{itemize}
	\item Vacuum fluctuation
	\begin{equation}
	\abr{\vbr{\what{f}}^2}_\text{vac} = \rfun{\Omicron}{q^{-1}}
	\end{equation}

	\item A black hole does \alert{not} alter the \alert{high}-energy
	processes
	\begin{equation}
	\abr{\vbr{\what{f}}^2}_{\chi_b} \approx
	\abr{\vbr{\what{f}}^2}_\text{th} \approx
	\abr{\vbr{\what{f}}^2}_\text{vac} = \rfun{\Omicron}{q^{-1}},
	\quad \vbr{p} \gg T_\text{HD}
	\end{equation}

	\item A black hole \alert{suppresses} low-energy fluctuation of the
	\alert{pure} state, while \alert{enhancing} that of the
	\alert{thermal} state
	\begin{equation}
	\rfun{\Omicron}{1} \sim \abr{\vbr{\what{f}}^2}_{\chi_b} \ll
	\abr{\vbr{\what{f}}^2}_\text{vac} \ll
	\abr{\vbr{\what{f}}^2}_\text{th} \sim \rfun{\Omicron}{q^{-2}},
	\quad \vbr{p} \ll T_\text{HD}
	\end{equation}
	\item Critical scale $\vbr{p} \sim T_\text{HD}$
\end{itemize}

\end{frame}

\section{Distance between Density Operators}

\begin{frame}{Trace Distance %\cite[ch.~9]{Wilde2009}
}{Definitions}
\begin{itemize}
\item Trace distance between Hermitian operators $\what{M}$ and $\what{N}$
\begin{equation}
\rfun{T}{\what{M}, \what{N}} \coloneqq \frac{1}{2} 
\tr \sqrt{\rbr{\what{M}-\what{N}}^\dagger\rbr{\what{M}-\what{N}}}
\label{eq:def-trace-dist}
\end{equation}
% \begin{itemize}
% \item Positivity, homogeneity, triangle ineq., isometric inv.
% \end{itemize}
\item For \alert{density operators} $\what{\rho}$ and $\what{\sigma}$,
\begin{itemize}
\item $0 \le \rfun{T}{\what{\rho}, \what{\sigma}} \le 1$;
\item Controlled by \alert{fidelity} in %\cite{Fuchs1999}
\begin{empheq}[box=\fbox]{equation}
1 - \rfun{F}{\what{\rho},\what{\sigma}} \le \rfun{T}{\what{\rho},\what{\sigma}}
\le \sqrt{1 - \rfun{F^2}{\what{\rho},\what{\sigma}}},
\label{eq:ineq-fvdg}
\end{empheq}
where we only need
\begin{equation}
\rfun{F}{\Ket{\alpha},\what{\sigma}} \coloneqq \Braket{\alpha | 
\what{\sigma} | \alpha}^\frac{1}{2}.
\label{eq:fidelity-pure-mixed}
\end{equation}
\item In our application, $T$ is difficult while $F$ can be obtained.
\end{itemize}
\end{itemize}
\end{frame}

\begin{frame}[allowframebreaks]{Single-mode Distances between the Density 
Operators}{Bounds set by fidelity and \cref{eq:ineq-fvdg}}

\begin{itemize}

%\item $\sfun{\chi}{f} \sim S$, $\sfun{\Psi}{g, \phi, f} \sim U$

\item The pure state can be decomposed upon discretisation 
$\sfun{\chi_b}{f} \sim \sum_p \rfun{\chi_b^\rbr{p}}{f_p} \equiv \sum_p 
\Braket{f_p | \chi_b^{\rbr{p}}}$, where $f_p \coloneqq \rfun{f}{p}$

\item So is the thermal density operator $\rfun{\what{\rho}_\text{th}}{T} \sim
\bigotimes_p \rfun{\what{\rho}_\text{th}^\rbr{p}}{T}$

\item Fidelity (in \cref{eq:ineq-fvdg}) can be factorised as well
\begin{equation}
F \equiv \Braket{\chi_b | \what{\rho}_\text{th} | \chi_b}^\frac{1}{2} \sim
\prod_p \Braket{\chi_b^{\rbr{p}} | \what{\rho}_\text{th}^{\rbr{p}}
| \chi_b^{\rbr{p}}}^\frac{1}{2} \eqqcolon \prod_p F^{\rbr{p}};
\end{equation}
\item $F^{\rbr{p}}$ can be computed \alert{in order to find bounds of $T$}
\begin{equation}
\rfun{F^{\rbr{p}}}{\Ket{p},\rfun{\what{\rho}_\text{th}^\rbr{p}}{T_\text{HD}}}
= \frac{\sqrt{u-1}}{\sqrt[4]{u^2+u+1}},\quad
u \coloneqq \ee^q \equiv \ee^{\vbr{p}/T_\text{HD}}.
\label{eq:fidelity-pt}
\end{equation}
\end{itemize}

%\begin{center}
%\input{./graphics/graph_trace_dist_mode_bounds_color}
%\end{center}
%Difference goes exponentially small w.r.t.\ $p$ or $\rbr{\text{wave 
%length}}^{-1}$

\end{frame}


\begin{frame}[allowframebreaks]{All-modes Distances between the Density 
Operators}{Bounds set by fidelity and \cref{eq:ineq-fvdg}}

\begin{itemize}

\item `Go to the continuum limit': $\Lambda$ dimension regulator
\begin{equation}
\sum_p \rfun{g}{p} \to \frac{1}{2\pp\Lambda} \int \dd p\,\rfun{g}{p},
\end{equation}

\item Analogously, to regularise a product
\begin{equation}
\prod_p \rfun{f}{p} \equiv \cfun{\exp}{\sum_p 
\ln\rfun{f}{p}} \to \cfun{\exp}{\frac{1}{2\pp\Lambda}
\int \dd p\,\ln\rfun{f}{p}}
\end{equation}

\item Regularised $F$ can be calculated \alert{in order to set bounds of $T$}
\begin{equation}
\rfun{F}{\chi_b, \what{\rho}_\text{th}} = 
\cfun{\exp}{\frac{2}{2\pp\Lambda}\int_0^{+\infty}\dd p\,\ln F^{\rbr{p}}} =
\rfun{\exp}{-\frac{\pp}{9} \frac{T_\text{HD}}{\Lambda}}
\end{equation}
\end{itemize}


%\begin{center}
%\input{./graphics/graph_trace_dist_total_bounds_color}
%\end{center}
%Difference goes exponentially small w.r.t.\ $T_\text{HD}$ or $M^{-1}$

\end{frame}



\section*{Summary}

\begin{frame}{Summary}

  % Keep the summary *very short*.
\begin{itemize}
\item
Compared the \alert{pure} and the \alert{thermal} descriptions of the
radiation within the solvable dilaton gravity model
\item
Fourier-mode fluctuation: that of the thermal state \alert{diverges} faster
than the vacuum case does at low energy, while the pure-state fluc.\ remains
\alert{finite}; at high energies they \alert{converge}.
\item
Trace distance: goes exponentially \alert{small} with black hole 
\alert{temperature} going to zero.
\end{itemize}
  
  % The following outlook is optional.
  \vskip0pt plus.5fill
  \begin{itemize}
  \item
	Outlook in the proposed PhD study
	\begin{itemize}
	\item Further discussion within full CGHS, BTZ, etc.
	\item Make use of the decoherence theory
	\end{itemize}
	\begin{itemize}
	\item Nature of the microscopic degrees of freedom for BH entropy
	\item Breakdown of the semi-classical approximations in QG
    \end{itemize}
%	\item Understand the dilaton gravity model
%	\item Understand the Fourier-mode fluctuation
%	\item Evaluate the real space correlator
%	\item Understand practical meaning of trace distance
%	\item Understand the regulator in total trace distance
%	\item Evaluate the exact trace distance
%	\item Include the gravitational wave functional
  \end{itemize}
\end{frame}



% All of the following is optional and typically not needed. 
\appendix
\section<presentation>*{\appendixname}
\subsection<presentation>*{For Further Reading}

\begin{frame}[allowframebreaks]
  \frametitle<presentation>{For Further Reading}
    
%  \begin{thebibliography}{10}
    
  \beamertemplatebookbibitems
  % Start with overview books.
\printbibliography[type=book]

%  \bibitem{Author1990}
%    A.~Author.
%    \newblock {\em Handbook of Everything}.
%    \newblock Some Press, 1990.
 
    
  \beamertemplatearticlebibitems
  % Followed by interesting articles. Keep the list short. 
\printbibliography[nottype=book]

%  \bibitem{Someone2000}
%    S.~Someone.
%    \newblock On this and that.
%    \newblock {\em Journal of This and That}, 2(1):50--100,
%    2000.
%  \end{thebibliography}
\end{frame}

\section<presentation>*{More on Trace Distance and Fidelity}

\begin{frame}{More on Trace Distance %\cite[ch.~9]{Wilde2009}
}{Interpretation}
\begin{itemize}
\item Maximal probability-difference obtainable
\begin{equation}
\rfun{T}{\what{\rho}, \what{\sigma}} = \max_{0 \le \what{\Lambda} \le 
\what{1}}
\cfun{\tr}{\what{\Lambda}\rbr{\what{\rho}-\what{\sigma}}},
\end{equation}
where all eigenvalues of $\what{\Lambda}$ are in the range $\sbr{0,1}$
\item E.g.\ $\what{\Lambda} \coloneqq \Ket{\alpha}\Bra{\alpha}$, 
$\Ket{\alpha}$
eigenstate of $\what{\Alpha}$ with eigenvalue $\alpha$
\begin{itemize}
\item $\cfun{\tr}{\what{\Lambda}\what{\rho}}$: the probability of getting
$\alpha$ in measuring $\what{\Alpha}$
\item $\cfun{\tr}{\what{\Lambda}\rbr{\what{\rho}-\what{\sigma}}}$: the
difference of the probability above
\item $\rfun{T}{\what{\rho}, \what{\sigma}}$: the maximal value of the 
difference above
\end{itemize}
\end{itemize}

\end{frame}




\begin{frame}{More on Fidelity}
\begin{itemize}
\item General definition %\cite[ch.~6]{Petz2008}
\begin{align}
\rfun{F}{\Ket{\alpha},\Ket{\beta}} &= \vbr{\Braket{\alpha | \beta }} \\
\rfun{F}{\Ket{\alpha},\what{\sigma}} &= \sqrt{\Braket{\alpha | \what{\sigma} | 
\alpha}}
\tag{\ref{eq:fidelity-pure-mixed} rev.}\\
\rfun{F}{\what{\rho},\what{\sigma}} &= \tr\sqrt{\what{\rho}^\frac{1}{2} 
\what{\sigma} \what{\rho}^\frac{1}{2}}
\end{align}
\item Intepretation: faithfulness
\begin{equation}
\rfun{F}{\Ket{\alpha}, \Ket{\alpha}} = 1
\end{equation}

\end{itemize}

\end{frame}

\section<presentation>*{Foundation of Statistical Physics}

\begin{frame}{Another New Foundation of Statistical Physics 
%\cite{Popescu2006}
}{Specific and easy version of the construction}

\begin{itemize}
\item Total isolated system $U$ with \alert{energy} constraint 
$\abr{\what{H}_U} \coloneqq E_U$, divided into a \alert{(sub)system} $S$ and 
an environment $E$

\item Hilbert spaces $\mscrH_R \supseteq \mscrH_U = \mscrH_S \otimes \mscrH_E$;
$\what{1}_R$ identity on $\mscrH_R$, dimension $d_R \coloneqq \dim \mscrH_R < 
+\infty$

\item Equiprobable / maximal-ignorant state of $U$
\begin{equation}
\what{\mscrE}_R \coloneqq d_R^{-1} \what{1}_R \in \mscrH_R
\end{equation}

\item Hamiltonians $\what{H}_U = \what{H}_S + \what{H}_E + \what{H}_\text{int}$

\item \alert{Canonical state} of $S$ with energy 
constraint %\cite[§~28]{Landau1980}
\begin{equation}
\what{\Omega}_S^\rbr{\text{E}} \coloneqq \tr_E \what{\mscrE}_R 
\propto\rfun{\exp}{- 
\what{H}_S/T_\text{th}}
\end{equation}

\item \alert{Theorem}: $\forall \Ket{\phi} \in \mscrH_R$, the reduced state of 
$S$
\begin{empheq}[box=\fbox]{equation}
\tr_E \Ket{\phi}\Bra{\phi} \eqqcolon \rfun{\what{\rho}_S}{\phi} \approx 
\what{\Omega}_S^\rbr{\text{E}}.
\end{empheq}

\end{itemize}

\end{frame}


\begin{frame}{Another New Foundation of Statistical Physics 
%\cite{Popescu2006}
}{Generic and exact version of the construction}

\begin{itemize}
\item \alert{Arbitrary} constraint $R$; study the \alert{trace 
distance} $T$ between $\rfun{\what{\rho}_S}{\phi}$ and $\what{\Omega}_S$

\item Lemma: \alert{average distance} is small w.r.t.\ $d_S/d_E^\text{eff}$
\begin{equation}
\abr{\rfun{T}{\rfun{\what{\rho}_S}{\phi}, \what{\Omega}_S}} \le 
\tfrac{1}{2}\sqrt{d_S/d_E^\text{eff}}
\end{equation}

\item Theorem: \alert{probability of large deviation} is exponentially 
small w.r.t.\ the distance; an easy version
\begin{equation}
\frac{\sfun{V}{\cbr{\Ket{\phi} \,\Big |\,
\rfun{T}{\rfun{\what{\rho}_S}{\phi}, \what{\Omega}_S} \geq 
d_R^{-\frac{1}{3}}}}}
{\sfun{V}{\cbr{\Ket{\phi} }}} \leq
4\rfun{\exp}{- \frac{2d_R^{\frac{1}{3}}}{9\pp^3} }
\end{equation}

\item Effective dimension of $E$: setting
$\what{\Omega}_E = \tr_S \what{\mscrE}_R$,
\begin{equation}
d_U/d_S \equiv d_E \ge d_E^\text{eff}
\coloneqq \rbr{\tr \what{\Omega}_E^2}^{-1} \ge d_R/d_S.
\end{equation}

\end{itemize}

\end{frame}


\end{document}


